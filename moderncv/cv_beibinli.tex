%%%%%%%%%%%%%%%%%%%%%%%%%%%%%%%%%%%%%%%%%
% "ModernCV" CV and Cover Letter
% LaTeX Template
% Version 1.2 (25/3/16)
% XeLaTeX
%
% This template has been downloaded from:
% http://www.LaTeXTemplates.com
%
% Original author:
%% Xavier Danaux (xdanaux@gmail.com) with modifications by:
% Vel (vel@latextemplates.com)
%
% License:
% CC BY-NC-SA 3.0 (http://creativecommons.org/licenses/by-nc-sa/3.0/)
%
% Important note:
% This template requires the moderncv.cls and .sty files to be in the same 
% directory as this .tex file. These files provide the resume style and themes 
% used for structuring the document.
%
%%%%%%%%%%%%%%%%%%%%%%%%%%%%%%%%%%%%%%%%%



%----------------------------------------------------------------------------------------
%	PACKAGES AND OTHER DOCUMENT CONFIGURATIONS
%----------------------------------------------------------------------------------------

\documentclass[11pt,letterpaper,roman]{moderncv} % Font sizes: 10, 11, or 12; paper sizes: a4paper, letterpaper, a5paper, legalpaper, executivepaper or landscape; font families: sans or roman

\moderncvstyle{classic} % CV theme - options include: 'casual' (default), 'classic', 'oldstyle' and 'banking'
\moderncvcolor{blue} % CV color - options include: 'blue' (default), 'orange', 'green', 'red', 'purple', 'grey' and 'black'

\usepackage{lipsum} % Used for inserting dummy 'Lorem ipsum' text into the template

\usepackage[scale=0.75]{geometry} % Reduce document margins
%\setlength{\hintscolumnwidth}{3cm} % Uncomment to change the width of the dates column
%\setlength{\makecvtitlenamewidth}{10cm} % For the 'classic' style, uncomment to adjust the width of the space allocated to your name

%----------------------------------------------------------------------------------------
%	NAME AND CONTACT INFORMATION SECTION
%----------------------------------------------------------------------------------------

\firstname{Beibin} % Your first name
\familyname{Li} % Your last name

% All information in this block is optional, comment out any lines you don't need
\title{Curriculum Vitae}
\address{SCRI-W8, 1900 9th Ave.}{Seattle, WA, 98101}
\mobile{(901) 734-3790}
%% \phone{(000) 111 1112}
%% \fax{(000) 111 1113}
\email{beibin.li@seattlechildrens.org}
\homepage{beibinli.com}{beibinli.com} % The first argument is the url for the clickable link, the second argument is the url displayed in the template - this allows special characters to be displayed such as the tilde in this example
%% \extrainfo{additional information}
% \photo[70pt][0.4pt]{pictures/summer} % The first bracket is the picture height, the second is the thickness of the frame around the picture (0pt for no frame)
%% \quote{"A witty and playful quotation" - John Smith}



%----------------------------------------------------------------------------------------

\begin{document}

%----------------------------------------------------------------------------------------
%	COVER LETTER
%----------------------------------------------------------------------------------------

% To remove the cover letter, comment out this entire block
%% 
%% \clearpage
%% 
%% \recipient{HR Department}{Corporation\\123 Pleasant Lane\\12345 City, State} % Letter recipient
%% \date{\today} % Letter date
%% \opening{Dear Sir or Madam,} % Opening greeting
%% \closing{Sincerely yours,} % Closing phrase
%% \enclosure[Attached]{curriculum vit\ae{}} % List of enclosed documents
%% 
%% \makelettertitle % Print letter title
%% 
%% \lipsum[1-3] % Dummy text
%% 
%% \makeletterclosing % Print letter signature
%% 
%% \newpage
%% 
%----------------------------------------------------------------------------------------
%	CURRICULUM VITAE
%----------------------------------------------------------------------------------------

\makecvtitle % Print the CV title

%----------------------------------------------------------------------------------------
%	EDUCATION SECTION
%----------------------------------------------------------------------------------------

\section{Education}

\cventry{May 2015}{Bachelor of Science}{Mathematics}{University of Michigan}{Ann Arbor}{}{}  % Arguments not required can be left empty
\cventry{May 2015}{Bachelor of Science}{Computer Science}{University of Michigan}{Ann Arbor}{}{}  % Arguments not required can be left empty
%% \cventry{2007--2010}{Bachelor of Business Studies}{The University of California}{Berkeley}{\textit{GPA -- 7.5}}{Specialized in Commerce}

%% \section{Masters Thesis}
%% \cvitem{Title}{\emph{Money Is The Root Of All Evil -- Or Is It?}}
%% \cvitem{Supervisors}{Professor James Smith \& Associate Professor Jane Smith}
%% \cvitem{Description}{This thesis explored the idea that money has been the cause of untold anguish and suffering in the world. I found that it has, in fact, not.}

%----------------------------------------------------------------------------------------
%	WORK EXPERIENCE SECTION
%----------------------------------------------------------------------------------------

\section{Experience}

%% \subsection{Vocational}

\cventry{2016--Present}{Research Associate in Computer Imaging}{\newline{}
\textsc{Seattle Children's Innovation and Technology Lab}}
{\newline{}Seattle Children's Research Institute}{}
{
Advisor: Frederick Shic, Ph.D.
}

\cventry{2015--2016}{Research Fellow in Translational Technologies in Development}
{\newline{}\textsc{Technology Innovation Laboratory}}{Child Study Center, Yale University}{}
{
Advisor: Frederick Shic, Ph.D.
\newline{}
Design eye-tracking experiments and systems using Presentation, Python, PsychoPy, SR EyeLink, Eye Tribe, and Arduino for children with Autism Spectrum Disorder (ASD). 
Design fixation identification algorithms for eye tracking technology, and use C++, Matlab, Python, and R to conduct post-hoc experiment data analysis.
Communicate with collaborating implementation sites to troubleshoot eye-tracking experiments in a large NIH-funded multi-site project.
Deploy and tune machine learning algorithms on robots to improve human robot interaction for children.
Implement virtual reality project using Python and Unity for Oculus Rift. 
}
%% 	Developed spreadsheets for risk analysis on exotic derivatives on a wide array of commodities (ags, oils, precious and base metals), managed blotter and secondary trades on structured notes, liaised with Middle Office, Sales and Structuring for bookkeeping.
%% \newline{}\newline{}
%% Detailed achievements:
%% \begin{itemize}
%% \item Learned how to make amazing coffee
%% \item Finally determined the reason for \textsc{PC LOAD LETTER}:
%% \begin{itemize}
%% \item Paper jam
%% \item Software issues:
%% \begin{itemize}
%% \item Word not sending the correct data to printer
%% \item Windows trying to print in letter format
%% \end{itemize}
%% \item Coffee spilled inside printer
%% \end{itemize}
%% \item Broke the office record for number of kitten pictures in cubicle
%% \end{itemize}}

%------------------------------------------------

\cventry{2014--2015}{Instructional Aide}{School of Engineering}{\newline{}\textsc{University of Michigan}}{Ann Arbor}{ Professors:  Seth Pettie, Ph.D., and Grant Schoenebeck, Ph.D.
\newline{}
EECS 376 (Foundations of Computer Science). 
\newline{}
Taught discussion sections on Finite Automata, Context Free Language, Turing Machine, complexity analysis, and NP problems. 
Answered students' questions in online forum and during office hours. Designed section notes, homework, exams, 
and graded exams for more than 300 students. 
Reviews from students: 
\textit{"Discussions are helpful. If the lectures were taught like the discussions, I would be getting a lot more out of this course"},
\textit{"...you answer my questions so well. You always seem to understand what the student is asking..."}
}

%------------------------------------------------
\cventry{2014--2015}{Research Fellow}{Transportation Research Institution}{\newline{}\textsc{University of Michigan}}{Ann Arbor}{ Professor: Paul Green, Ph.D.
\newline{}
Used ISAT to design virtual roads for a driving recognition system experiment. 
Used JMP and R to analyze data from transportation research experiments.
Taught ergometrics students to use software: Morae, Cogtool, and IMPRINT to design user-friendly interface.
}

%------------------------------------------------
%% \subsection{Miscellaneous}
%% \cventry{2008--2009}{Computer Repair Specialist}{Buy More}{Burbank}{}{Worked in the Nerd Herd and helped to solve computer problems by asking customers to turn their computers off and on again.}

%----------------------------------------------------------------------------------------
%	AWARDS SECTION
%----------------------------------------------------------------------------------------

\section{Awards}

\cvitem{2014}{The Mathematical Contest in Modeling (MCM), Honorable Mention}
\cvitem{2013--2015}{University Honor, University of Michigan}
\cvitem{2010}{Presidential Scholarship, Rhodes College}

%----------------------------------------------------------------------------------------
%	PUBLICATIONS SECTION
%----------------------------------------------------------------------------------------
\section{Publications}


\cvitem{2016}{ \textbf{Li, B.}, Wang, Q., Barney, E., Hart, L., Wall, C., Chawarska, K., ... \& Shic, F. (2016, March). Modified DBSCAN algorithm on oculomotor fixation identification. In 
\textit{Proceedings of the Ninth Biennial ACM Symposium on Eye Tracking Research \& Applications} (pp. 337-338). ACM.}


% \cvitem{2016}{ \textbf{Beibin Li}, Quan Wang, Erin Barney, Logan Hart, Carla Wall, Katarzyna Chawarska, Irati Saez de Urabain, Timothy J. Smith, and Frederick Shic, "Modified DBSCAN Algorithm on Oculomotor Fixation Identification", \textit{ACM Eye-Tracking Research and Applications Symposium 2016 (ETRA 2016)} }


% \cvitem{2016}{ \textbf{Beibin Li}, Quan Wang, Laura Boccanfuso, and Frederick Shic, "Optimality of the Distance Dispersion Fixation Identification Algorithm", \textit{ACM Eye-Tracking Research and Applications Symposium 2016 (ETRA 2016)} }

\cvitem{2016}{ \textbf{Li, B.}, Wang, Q., Boccanfuso, L., \& Shic, F. (2016, March). Optimality of the distance dispersion fixation identification algorithm. In 
\textit{Proceedings of the Ninth Biennial ACM Symposium on Eye Tracking Research \& Applications} (pp. 339-340). ACM.}


\cvitem{2016}{ Wang, Q., Boccanfuso, L., \textbf{Li, B.}, Ahn, A. Y. J., Foster, C. E., Orr, M. P., ... \& Shic, F. (2016, March). Thermographic eye tracking. In
\textit{Proceedings of the Ninth Biennial ACM Symposium on Eye Tracking Research \& Applications} (pp. 307-310). ACM.}


% \cvitem{2016}{Quan Wang, Laura Boccanfuso, \textbf{Beibin Li}, Amy Yeo-jin Ahn, Claire E. Foster, Margaret P. Orr, Brian Scassellati, Frederick Shic, "Thermographic Eye Tracking", \textit{ACM Eye-Tracking Research and Applications Symposium 2016 (ETRA 2016)} }

\cvitem{2016}{ Boccanfuso, L., Wang, Q., Leite, I., \textbf{Li, B.}, Torres, C., Chen, L., Salomons, N., Foster, C., Barney, E., Ahn, Y., Scassellati, B., \& Shic, F.. A Thermal Emotion Classifier for Improved Human-Robot Interaction.
\textit{IEEE International Symposium on Robot and Human Interactive Communication 2016 (RO-MAN 2016).}
}

\cvitem{2016}{ \textbf{Li, B.}, Boccanfuso, L., Wang, Q., \& Shic, F.. 
Human Robot Activity Classification based on Accelerometer and Gyroscope.
\textit{IEEE International Symposium on Robot and Human Interactive Communication 2016 (RO-MAN 2016).}
}


%----------------------------------------------------------------------------------------
%	PRESENTATION SECTION
%----------------------------------------------------------------------------------------
\section{Presentations}
\cvitem{2016}{ \textbf{Li, B.}
(2016, July. 8).
Human Robot Activity Classification for Children with Autism.
% \textit{}.
Child Study Center, Yale University, New Haven, CT.
}

\cvitem{2016}{ \textbf{Li, B.}
(2016, Feb. 26).
Low Cost and Portable Eye Tracker.
% \textit{}.
Center For Children With Special Needs, Glastonbury, CT.
}

\cvitem{2015}{ \textbf{Li, B.}
% Should I put collaborators name here? Even if I am the only presenter % 
% \cvitem{2015}{ \textbf{Li, B.}, Boccanfuso, L.,  Valencia, S., \& Shic, F.
(2015, Nov. 7).
Background Music and Sound Effects in Human-Robot Interaction.
\textit{Northeast Robotics Colloquium 2015}.
Worcester Polytechnic Institute, Worcester, MA.
}


%----------------------------------------------------------------------------------------
%	CURRENT PROJECTS SECTION
%----------------------------------------------------------------------------------------
\section{Current Projects}

\cventry{2015--Present}{NIH U19 MH108206-01}{\newline{}\textsc{The Autism Biomarkers Consortium for Clinical Trials}}{\newline{}PI: McPartland, James}{}{
% Role: Researcher \newline{}
Helped create eye-tracking experiments using Neurobs Presentation software, and improved eye-tracking calibration protocol. 
Designed a 73Hz system to measure and record light condition using Arduino and TSL2561 sensor. 
Built and set up eye-tracking system with SR EyeLink 1000 Plus eye trackers, web-cams, DVD recorders, and light meters.
Analyze pupillary light reflex and other eye-tracking experiment data from children.
Process and analyze 500 Hz eye tracking data collected across other sites. 
Troubleshoot eye tracking experiment and analysis across five sites, including Yale University, Boston Children's Hospital, University of Washington/Seattle Children's Research Institute, University of California (Los Angeles), and Duke University. 
}


\cventry{2015--Present}{Simons Foundation 15-004376}{\newline{}\textsc{Tracking Intervention Effects with Eye Tracking}}{\newline{}PI: Shic, Frederick, Ph.D.}{}{
% Role: Researcher \newline{}
Helped design experiments and counterbalance eye-tracking stimuli. Built and tested eye-tracking experiments using SR eye tracker. Analyzed iPad eye-tracking data using Cambridge face tracker (CLM-framework) and OpenCV in Matlab.
}


\cventry{2015--Present}{Hebrew University Eye-Tracking Project}{}{}{}{
% Role: Researcher \newline{}
Used PsychoPy to design eye-tracking experiments and Eye Tribe to collect data in Israel. 
Filtered and analyzed experimental data with Python and R. This project deploys portable eye-tracking experiments for children with ASD outside the United States.
}


%----------------------------------------------------------------------------------------
%	PAST PROJECTS SECTION
%----------------------------------------------------------------------------------------
\section{Past Projects}

\cventry{May 2015}{StagePlay}{Swift}{}{}{
Designed an iOS application for actors to practice their lines and to collaborate with their partners.
Main features: line-by-line display, performance recording, and script editing. Compatible with iPhone and iPad.
}


\cventry{Feb. 2015}{Course Scheduler}{C++}{}{}{
Completed back-end website design for students to schedule the following year's courses. Designed and implemented algorithms in PHP, and imported 10,000 courses into SQL database. Coordinated with front-end developers.
}


\cventry{Oct. - Dec. 2014}{Medieval World Game}{C++}{}{}{
Developed a command line game for creating different characters and buildings. Applied C++ idioms and design patterns (Model View Controller, Composite, factory, etc.) so new features could be added easily.
}


\cventry{Sept. - Oct. 2014}{Meeting Manager}{C++}{}{}{
Designed a meeting management command line software by using classes for abstraction and encapsulation. Implemented linked-lists, arrays, and strings that behaved like build-in types; used strong exception guarantees. Managed dynamically allocated memory with copy and move construction and assignment. 
}


\cventry{Mar. 2014}{Stock Exchange}{C++}{}{}{
Designed an electronic exchange simulator by using priority queue to store buyers' and sellers' bids. Stored stock information using customized Hash-Table.
}


%----------------------------------------------------------------------------------------
%	COMPUTER SKILLS SECTION
%----------------------------------------------------------------------------------------

\section{Computer Skills}

\cvitem{Advanced}{\textsc{C++}, \textsc{Python}, \textsc{Matlab}, \textsc{R}, \textsc{Vim} }
\cvitem{Intermediate}{\textsc{html}, \LaTeX, \textsc{Git}, Swift, SQL, Visual Studio, XCode, Eclipse, Mathematica }
\cvitem{Basic}{SPSS, JMP, Unity }


%----------------------------------------------------------------------------------------
%	membership
%----------------------------------------------------------------------------------------

\section{Membership}

\cvitem{}{Institute of Electrical and Electronics Engineers }
\cvitem{}{Association for Computing Machinery}
\cvitem{}{International Society for Autism Research}

%----------------------------------------------------------------------------------------
%	INTERESTS SECTION
%----------------------------------------------------------------------------------------

\section{Research Interests}

\renewcommand{\listitemsymbol}{-~} % Changes the symbol used for lists

\cvlistdoubleitem{Computer Vision}{Machine Learning}
\cvlistdoubleitem{Optimization}{Artificial Intelligence}

%----------------------------------------------------------------------------------------
%	references
%----------------------------------------------------------------------------------------

\section{References}

\cventry{}
{Frederick Shic}
{\newline{}Associate Professor}
{\newline{}University of Washington, and Seattle Children's Research Institute}{}
{
(206)884-8162
\newline{}fshic@uw.edu
\newline{}
}

\cventry{}
{Paul A. Green}
{\newline{}Professor}
{\newline{}University of Michigan, Ann Arbor}{}
{
(734)-763-37952
\newline{}pagreen@umich.edu
\newline{}
}



%----------------------------------------------------------------------------------------
%	COMMUNICATION SKILLS SECTION
%----------------------------------------------------------------------------------------

%% \section{Communication Skills}
%% 
%% \cvitem{2010}{Oral Presentation at the California Business Conference}
%% \cvitem{2009}{Poster at the Annual Business Conference in Oregon}

%----------------------------------------------------------------------------------------
%	LANGUAGES SECTION
%----------------------------------------------------------------------------------------

%% \section{Languages}
%% 
%% \cvitemwithcomment{English}{Mothertongue}{}
%% \cvitemwithcomment{Spanish}{Intermediate}{Conversationally fluent}
%% \cvitemwithcomment{Dutch}{Basic}{Basic words and phrases only}

%----------------------------------------------------------------------------------------

\end{document}
